\documentclass[10pt]{article}
\usepackage{amsmath}
\usepackage{listings}
\usepackage{graphicx}
\graphicspath{ {./images/} }
\begin{document}
{\centering
    Advanced Computer Linguistics - Assignment 1
    \par
    Samuel Petit - 17333946
    \par
}
\section*{Question 1}
(i) implies (ii):
\begin{equation*}
    P(A \cap B) = P(A|B) * P(B) = P(A) * P(B)
\end{equation*}
(ii) implies (i):
\begin{equation*}
    P(A|B) = \frac{P(A \cap B)}{P(B)} = \frac{P(A) * P(B)}{P(B)} = P(A)
\end{equation*}

\section*{Question 2}
\subsection*{a}
\begin{equation*}
    P(gw|ps) = \frac{28}{28 + 40} = \frac{28}{30}
\end{equation*}
Counts $gw  \cap\neg ps$ and $\neg gw  \cap\neg ps$ are not useful in this situation.

\subsection*{b}
\begin{equation*}
    P(ps|gw) = \frac{28}{28 + 2} = \frac{28}{168}
\end{equation*}
Counts $gw  \cap\neg ps$ and $\neg gw  \cap\neg ps$ are not useful in this situation.

\section*{Question 3}
\subsection*{a}
We have:
\begin{equation*}
    P(vmel) = 0.01
\end{equation*}
\begin{equation*}
    P(dbi|vmel) = 0.95
\end{equation*}
\begin{equation*}
    P(dbi|\neg vmel) = 0.01
\end{equation*}

Let's compute helper values for doing marginalization:
\begin{equation*}
    P(\neg vmel) = 1 - P(vmel) = 1 - 0.01 = 0.99
\end{equation*}
\begin{equation*}
    P(dbi \cap vmel) = P(dbi|vmel) * p(vmel) = 0.95 * 0.01 = 0.095
\end{equation*}
\begin{equation*}
    P(dbi \cap \neg vmel) = P(dbi|\neg vmel) * p(\neg vmel) = 0.01 * 0.99 = 0.099
\end{equation*}


We can now compute $P(vmel|dbi)$ and $P(\neg vmel|dbi)$:
\begin{equation*}
    P(vmel|dbi) = \frac{P(dbi \cap vmel)}{P(dbi \cap vmel) + P(dbi \cap \neg vmel)}
    = \frac{0.095}{0.095 + 0.099} = \frac{95}{194}
\end{equation*}
\begin{equation*}
    P(\neg vmel|dbi) = \frac{P(dbi \cap \neg vmel)}{P(dbi \cap vmel) + P(dbi \cap \neg vmel)}
    = \frac{0.099}{0.095 + 0.099} = \frac{99}{194}
\end{equation*}


We find that $P(vmel|dbi) < P(\neg vmel|dbi)$ so $\neg vmel$ is the
best guess.


\subsection*{b}
We have:
\begin{equation*}
    P(vmel) = 0.15
\end{equation*}
\begin{equation*}
    P(dbi|vmel) = 0.95
\end{equation*}
\begin{equation*}
    P(dbi|\neg vmel) = 0.01
\end{equation*}

Let's compute helper values for doing marginalization:
\begin{equation*}
    P(\neg vmel) = 1 - P(vmel) = 1 - 0.15 = 0.85
\end{equation*}
\begin{equation*}
    P(dbi \cap vmel) = P(dbi|vmel) * p(vmel) = 0.95 * 0.15 = 0.1425
\end{equation*}
\begin{equation*}
    P(dbi \cap \neg vmel) = P(dbi|\neg vmel) * p(\neg vmel) = 0.01 * 0.85 = 0.0085
\end{equation*}


We can now compute $P(vmel|dbi)$ and $P(\neg vmel|dbi)$:
\begin{equation*}
    P(vmel|dbi) = \frac{P(dbi \cap vmel)}{P(dbi \cap vmel) + P(dbi \cap \neg vmel)}
    = \frac{0.1425}{0.1425 + 0.0085} = \frac{1425}{1510} = \frac{285}{302}
\end{equation*}
\begin{equation*}
    P(\neg vmel|dbi) = \frac{P(dbi \cap \neg vmel)}{P(dbi \cap vmel) + P(dbi \cap \neg vmel)}
    = \frac{0.0085}{0.1425 + 0.0085} = \frac{85}{1510} = \frac{17}{302}
\end{equation*}

We find that $P(vmel|dbi) > P(\neg vmel|dbi)$ so $vmel$ is the
best guess.

\subsection*{c}
We have:
\begin{equation*}
    P(vmel) = 0.01
\end{equation*}
\begin{equation*}
    P(dbi|vmel) = 0.95
\end{equation*}
\begin{equation*}
    P(dbi|\neg vmel) = 0.001
\end{equation*}

Let's compute helper values for doing marginalization:
\begin{equation*}
    P(\neg vmel) = 1 - P(vmel) = 1 - 0.01 = 0.99
\end{equation*}
\begin{equation*}
    P(dbi \cap vmel) = P(dbi|vmel) * p(vmel) = 0.95 * 0.01 = 0.0095
\end{equation*}
\begin{equation*}
    P(dbi \cap \neg vmel) = P(dbi|\neg vmel) * p(\neg vmel) = 0.001 * 0.99 = 0.00099
\end{equation*}


We can now compute $P(vmel|dbi)$ and $P(\neg vmel|dbi)$:
\begin{equation*}
    P(vmel|dbi) = \frac{P(dbi \cap vmel)}{P(dbi \cap vmel) + P(dbi \cap \neg vmel)}
    = \frac{0.0095}{0.0095 + 0.00099} = \frac{950}{1049}
\end{equation*}
\begin{equation*}
    P(\neg vmel|dbi) = \frac{P(dbi \cap \neg vmel)}{P(dbi \cap vmel) + P(dbi \cap \neg vmel)}
    = \frac{0.00099}{0.0095 + 0.00099} = \frac{99}{1049}
\end{equation*}

We find that $P(vmel|dbi) > P(\neg vmel|dbi)$ so $vmel$ is the
best guess.

\section*{Question 4}
\begin{equation}
    P(cool: +) = \frac{62 + 108}{38 + 292} = \frac{170}{330} = 0.52
\end{equation}
\begin{equation}
    P(cool: +|noisy: +) = \frac{62}{62 + 38} = \frac{62}{100} = 0.62
\end{equation}

We have $P(cool: +) \neq P(cool: +|noisy: +)$ so the variables are not independant.

\section*{Question 5}
\begin{equation}
    P(cool: +| open: +) = \frac{90}{100} = 0.9
\end{equation}
\begin{equation}
    P(cool: +| open: +, noisy: +) = \frac{54}{64} = 0.9
\end{equation}

We have $P(cool: +| open: +) = P(cool: +| open: +, noisy: +)$ so
cool: + is conditionally independant of noisy: + given open: +.


\begin{equation}
    P(cool: +| open: -) = \frac{80}{400} = \frac{8}{40}
\end{equation}
\begin{equation}
    P(cool: +| open: -, noisy: +) = \frac{8}{40}
\end{equation}

We have $P(cool: +| open: -) = P(cool: +| open: -, noisy: +)$ so
cool: + is conditionally independant of noisy: + given open: -.


\end{document}
